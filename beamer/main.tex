% !TEX program = xelatex
% !TEX options = --shell-escape -synctex=1 -interaction=nonstopmode -file-line-error "%DOC%"
% !BIB program = bibtex
\documentclass[handout,10pt]{beamer}
% \documentclass{beamer}

\usetheme{Madrid}
\usecolortheme{default}

\usepackage{ctex}
\usepackage{listings}
\usepackage{multicol}
\usepackage{multirow}
\usepackage{verbatim}
\usepackage{makecell}
\usepackage{xcolor}
\usepackage{svg}
\usepackage{booktabs}

\usepackage[detect-all,locale=DE]{siunitx}
\usepackage{beamerthemeTsinghua}
\usefonttheme[onlymath]{serif}

\newcommand{\ud}{\mathrm{d}}
\newcommand{\mev}{\mathrm{MeV}}
\newcommand{\gev}{\mathrm{GeV}}
% \setbeamercolor{structure}{fg=purple}

\title[Multigroup Diffusion Equations]{带反射层的双群扩散理论通量计算}
\author[Dacheng Xu]{徐大成}
\institute[DEP]{清华大学工程物理系}
\date{2022.06.09}

\AtBeginSection[]
{
    \begin{frame}[noframenumbering]
        \frametitle{Outline}
        \thispagestyle{empty}
        \tableofcontents[currentsection]
    \end{frame}
}

\setlength{\abovecaptionskip}{0.cm}
\begin{document}
\frame{\titlepage}

\begin{frame}[noframenumbering]
\frametitle{Outline}
\thispagestyle{empty}
\tableofcontents
\end{frame}

\section{矢量化双群扩散方程}

\begin{frame}
\frametitle{矢量化双群扩散方程}
无限大平板反应堆,芯部宽度$r_0$有限且中心位于坐标原点处,反射层为无限大
\begin{align*}
    \label{eq:two3_group}
    \begin{split}
        -D_1\nabla^2\phi_1(r) + \Sigma_r\phi_1(r) &= \frac{1}{k_\mathrm{eff}}\left[\nu\Sigma_{f1}\phi_1(r)+\nu\Sigma_{f2}\phi_2(r)\right] \\
        -D_2\nabla^2\phi_2(r) + \Sigma_{a2}\phi_2(r) &= \Sigma_{1\rightarrow 2}\phi_1(r)
    \end{split}
\end{align*}
\noindent\rule{\textwidth}{0.5pt}
\begin{columns}
\column{0.5\textwidth}
\begin{equation*}
    \label{eq:vector_phi}
    \pmb{\Phi} = \begin{bmatrix}
        \phi_1(r) \\
        \phi_2(r)
    \end{bmatrix}
\end{equation*}
\begin{equation*}
    \label{eq:A_vector_phi}
    \pmb{\nabla}^2\pmb{\Phi}(\pmb{r})+\pmb{A}\pmb{\Phi}(\pmb{r}) = \pmb{0}
\end{equation*}
\column{0.5\textwidth}
\begin{equation*}
    \label{eq:A}
    \pmb{A} = \begin{bmatrix}
        \frac{\frac{\nu\Sigma_{f1}}{\textcolor{red}{k_\mathrm{eff}}}-\Sigma_r}{D_1} & \frac{\nu\Sigma_{f2}}{\textcolor{red}{k_\mathrm{eff}}D_1} \\
        \frac{\Sigma_{1\rightarrow2}}{D_2} & -\frac{\Sigma_{a2}}{D_2}
    \end{bmatrix}
\end{equation*}
\end{columns}
\ 

\ 

其中$\pmb{A}$是$k_\mathrm{eff}$的函数
\end{frame}

\begin{frame}
\frametitle{芯部矩阵$\pmb{A}$特征值分解}
\footnotesize
\begin{columns}
\column{0.5\textwidth}
\begin{align*}
    \begin{split}
        \mu^2 &= \frac{1}{2}\left[-\left(\frac{1}{L^2}+\frac{1}{\tau}\right)+\sqrt{\left(\frac{1}{L^2}+\frac{1}{\tau}\right)^2+\frac{4(\delta-1)}{\tau L^2}}\right] \\
        \nu^2 &= \frac{1}{2}\left[\left(\frac{1}{L^2}+\frac{1}{\tau}\right)+\sqrt{\left(\frac{1}{L^2}+\frac{1}{\tau}\right)^2+\frac{4(\delta-1)}{\tau L^2}}\right]
    \end{split}
\end{align*}
\column{0.5\textwidth}
\begin{align*}
    \begin{split}
        L^2 = \frac{D_2}{\Sigma_{a2}},\ \tau &= \frac{D_1}{\Sigma_r - \frac{\nu\Sigma_{f1}}{k_\mathrm{eff}}},\ \delta = \frac{k'-\frac{\nu\Sigma_{f1}}{\Sigma_r}}{k_\mathrm{eff}-\frac{\nu\Sigma_{f1}}{\Sigma_r}} \\
        k' &= \frac{\nu\Sigma_{f1}+\nu\Sigma_{f2}\frac{\Sigma_{1\rightarrow 2}}{\Sigma_{a2}}}{\Sigma_r}
    \end{split}
\end{align*}
\end{columns}

\noindent\rule{\textwidth}{0.5pt}

\begin{columns}
\column{0.5\textwidth}
\begin{align*}
    \begin{split}
        \pmb{u}_1 = \begin{bmatrix}
            \alpha \\
            s_1\alpha
        \end{bmatrix},\ 
        \pmb{u}_2 = \begin{bmatrix}
            \beta \\
            s_2\beta
        \end{bmatrix}
    \end{split}
\end{align*}
\column{0.5\textwidth}
\begin{align*}
    s_1 &= \frac{\frac{\Sigma_{1\rightarrow 2}}{D_2}}{\mu^2 + \frac{1}{L^2}},\ s_2 = \frac{\frac{\Sigma_{1\rightarrow 2}}{D_2}}{\frac{1}{L^2}-\nu^2} \\
    \alpha &= \left(1+s_1^2\right)^{-1/2},\ \beta = \left(1+s_2^2\right)^{-1/2}
\end{align*}
\end{columns}
\noindent\rule{\textwidth}{0.5pt}
\begin{columns}
\column{0.5\textwidth}
\begin{align*}
    \pmb{U} &= \begin{bmatrix}
        \pmb{u}_1, \pmb{u}_2
    \end{bmatrix}
\end{align*}
\begin{align*}
    \pmb{A}\pmb{U} &= \pmb{A}\begin{bmatrix}
        \pmb{u}_1, \pmb{u}_2
    \end{bmatrix}= \pmb{U}\begin{bmatrix}
        \mu^2 & \\
         & -\nu^2
    \end{bmatrix}
\end{align*}
\column{0.5\textwidth}
\begin{equation*}
    \pmb{\Phi}(r) = \pmb{U}\pmb{\Psi}(r),\ 
    \pmb{\Psi} = \begin{bmatrix}
        \psi_1(r) \\
        \psi_2(r)
    \end{bmatrix}
\end{equation*}
\begin{equation*}
    \pmb{U}\left\{\pmb{\nabla}^2\pmb{\Psi}(r)+\pmb{\Lambda}\pmb{\Psi}(r)\right\} = \pmb{0}
\end{equation*}
\end{columns}
\normalsize
\end{frame}

\begin{frame}
\frametitle{双群扩散理论的群常数}
\begin{table}[H]
    \centering
    \caption{一个典型PWR堆芯的双群扩散理论的群常数}
    \label{tab:constants}
    \begin{tabular}{lllll}
    \toprule
    \multirow{2}{*}{Group constants} & \multicolumn{2}{c}{Core} & \multicolumn{2}{c}{Reflective layer} \\
    \cmidrule(lr){2-3}\cmidrule(lr){4-5}
     & Group 1 & Group 2 & Group 1 & Group 2 \\
    \midrule
    $\nu\Sigma_f/\si{cm^{-1}}$ & $0.008476$ & $0.18514$ & $0$ & $0$ \\
    $\Sigma_a/\si{cm^{-1}}$ & $0.01207$ & $0.1210$ & $0.0004$ & $0.0197$ \\
    $D/\si{cm}$ & $1.2627$ & $0.3543$ & $1.13$ & $0.16$ \\
    $\Sigma_r/\si{cm^{-1}}$ & $0.02619$ & $0.1210$ & $0.0494$ & $0.0197$ \\
    \bottomrule
    \end{tabular}
\end{table}
\end{frame}

\section{临界宽度求解}

\begin{frame}
\frametitle{芯部中子通量密度参数化}
\begin{align*}
    \begin{split}
        \psi_{1c}(r) &= A_c\cos{\mu_c r} \\
        \psi_{2c}(r) &= B_c\left(e^{-\nu_c r} + e^{\nu_c r}\right)
    \end{split}
\end{align*}
\begin{align*}
    \begin{split}
        \phi_{1c}(r) &= \alpha\psi_{1c}(r) + \beta\psi_{2c}(r) = \alpha A_c\cos{\mu_c r} + \beta B_c\left(e^{-\nu_c r} + e^{\nu_c r}\right) \\
        \phi_{2c}(r) &= s_1\alpha\psi_{1c}(r) + s_2\beta\psi_{2c}(r) = s_1\alpha A_c\cos{\mu_c r} + s_2\beta B_c\left(e^{-\nu_c r} + e^{\nu_c r}\right)
    \end{split}
\end{align*}
\end{frame}

\begin{frame}
\frametitle{反射层中子通量密度参数化}
\begin{align*}
    \begin{split}
        -D_{1r}\nabla^2\phi_{1r}(r) + \Sigma_{rr}\phi_{1r}(r) &= 0 \\
        -D_{2r}\nabla^2\phi_{2r}(r) + \Sigma_{a2r}\phi_{2r}(r) &= \Sigma_{1\rightarrow 2,r}\phi_{1r}(r)
    \end{split}
\end{align*}
\begin{align*}
    \begin{split}
        \mu_r &= \sqrt{\frac{\Sigma_{rr}}{D_{1r}}} \\
        \nu_r &= \sqrt{\frac{\Sigma_{a2r}}{D_{2r}}} \\
        \gamma_r &= \sqrt{\frac{\Sigma_{1\rightarrow 2,r}}{D_{2r}}}
    \end{split}
\end{align*}
\begin{align*}
    \begin{split}
        \phi_{1r}(r) &= A_r e^{-\mu_r r} \\
        \phi_{2r}(r) &= B_r e^{-\nu_r r} + \frac{\gamma_r^2}{\nu_r^2 - \mu_r^2}A_r e^{-\mu_r r}
    \end{split}
\end{align*}
\end{frame}

\begin{frame}
\frametitle{边界条件}
\begin{columns}
\column{0.5\textwidth}
\begin{align*}
    \begin{split}
        \phi_{jc}(r_0 / 2) &= \phi_{jr}(r_0 / 2) \\
        D_{jc}\phi_{jc}(r_0 / 2) &= D_{jr}\phi_{jr}(r_0 / 2) \\
        j = 1,2
    \end{split}
\end{align*}
\column{0.5\textwidth}
\begin{align*}
    \label{eq:M_AB}
    \pmb{M}(\textcolor{red}{r_0}) \cdot \begin{bmatrix}
        A_c \\
        B_c \\
        A_r \\
        B_r
    \end{bmatrix} &= \pmb{0}
\end{align*}
\end{columns}
\tiny
\begin{align*}
    \pmb{M}(\textcolor{red}{r_0}) &= 
    \begin{bmatrix}
        \alpha\cos(\frac{\mu_c r_0}{2}) & \beta(e^{-\frac{\nu_c r_0}{2}}+e^{\frac{\nu_c r_0}{2}}) & -e^{-\frac{\mu_r r_0}{2}} & 0 \\
        s_1\alpha\cos(\frac{\mu_c r_0}{2}) & s_2\beta(e^{-\frac{\nu_c r_0}{2}}+e^{\frac{\nu_c r_0}{2}}) & -\frac{\gamma_r^2}{\nu_r^2 - \mu_r^2}e^{-\frac{\mu_r r_0}{2}} & -e^{-\frac{\nu_r r_0}{2}} \\
        -D_{1c}\mu_c\alpha\cos(\frac{\mu_c r_0}{2}) & -D_{1c}\beta(-\nu_c e^{-\frac{\nu_c r_0}{2}}+\nu_c e^{\frac{\nu_c r_0}{2}}) & D_{1r}\mu_r e^{-\frac{\mu_r r_0}{2}} & 0 \\
        -D_{2c}\mu_c s_1\alpha\cos(\frac{\mu_c r_0}{2}) & -D_{2c}s_2\beta(-\nu_c e^{-\frac{\nu_c r_0}{2}}+\nu_c e^{\frac{\nu_c r_0}{2}}) & D_{2r}\mu_r\frac{\gamma_r^2}{\nu_r^2 - \mu_r^2}e^{-\frac{\mu_r r_0}{2}} & D_{2r}\nu_r e^{-\frac{\nu_r r_0}{2}}
    \end{bmatrix}
\end{align*}
\normalsize
现只需解方程:$\det \pmb{M}(r_0)=0$
\end{frame}

\begin{frame}
\frametitle{临界宽度求解}
解得临界宽度$r_0=38.715\si{cm}$
\begin{columns}
\column{0.5\textwidth}
\begin{figure}[H]
    \centering
    \includesvg[width=1.0\linewidth]{figures/det_M_r.svg}
    \caption{$\det \pmb{M}(r)$的参数扫描,红色圆点处为$\left(r_0,\det \pmb{M}(r_0)\right)$}
    \label{fig:det_M}
\end{figure}
\column{0.5\textwidth}
\begin{figure}[H]
    \centering
    \includesvg[width=1.0\linewidth]{figures/det_M_r_far.svg}
    \caption{$\det \pmb{M}(r)$的参数扫描,红色圆点处为$\det \pmb{M}(r_0)=0$的多个解}
    \label{fig:det_M_r_far}
\end{figure}
\end{columns}
\end{frame}

\section{双群中子通量与伴随通量}

\begin{frame}
\frametitle{双群中子通量密度}
令$A_c=1$,将$\pmb{M}$表示为分块矩阵:

\begin{columns}
\column{0.5\textwidth}
\begin{align*}
    \pmb{M} &= \begin{bmatrix}
        M_{00} & M_{0,1\rightarrow 3} \\
        M_{1\rightarrow 3,0} & M_{1\rightarrow 3,1\rightarrow 3} \\
    \end{bmatrix}
\end{align*}
\column{0.5\textwidth}
\begin{align*}
    \begin{bmatrix}
        B_c \\
        A_r \\
        B_r
    \end{bmatrix} &= 
    -M_{1\rightarrow 3,1\rightarrow 3}^{-1}M_{1\rightarrow 3,0}
\end{align*}
\end{columns}
以求得的$\left[A_c,B_c,A_r,B_r\right]^{\mathrm{T}}$计算$\phi_1,\phi_2$,得到双群中子通量密度
\end{frame}

\begin{frame}
\frametitle{双群中子通量密度与伴随通量密度}
\begin{columns}
\column{0.5\textwidth}
\begin{figure}[H]
    \centering
    \includesvg[width=1.0\linewidth]{figures/flux_r.svg}
    \caption{$\phi_1,\phi_2$计算结果,红色虚线为芯部与反射层的边界}
    \label{fig:flux_r}
\end{figure}
\column{0.5\textwidth}
\begin{figure}[H]
    \centering
    \includesvg[width=1.0\linewidth]{figures/flux_r_adjoint.svg}
    \caption{$\phi_1^*,\phi_2^*$计算结果,红色虚线为芯部与反射层的边界}
    \label{fig:flux_r_adjoint}
\end{figure}
\end{columns}
\end{frame}

\section{讨论与总结}

\begin{frame}
\frametitle{讨论与总结}
\begin{itemize}
    \item 中子通量密度曲线在芯部与反射层交界面附近有一个凸起,主要原因是反射层内的热中子吸收比芯部小,而其慢化能力比芯部强,由堆芯泄漏出来的快中子,由于反射层的慢化作用在反射层内大部分都变成了热中子。
    \item 快中子伴随通量密度$\phi^*_{1c}$要比热中子伴随通量密度$\phi^*_{2c}$小很多,说明对芯部引入慢中子吸收剂将引起更大的反应性变化,其中子价值更高。
    \item 从双群扩散方程出发,以一个典型压水堆的群常数为例,计算了双群中子通量密度。计算结果符合物理预期,加深了作者对反应堆物理的认知。
\end{itemize}
\end{frame}

\appendix
\begin{frame}[allowframebreaks,noframenumbering]
\frametitle{双群伴随通量的$\pmb{M}(r_0)$}
\tiny
\begin{align*}
    \pmb{M}^*(r_0) &= \\ 
    &\begin{bmatrix}
        \alpha\cos(\frac{\mu_c r_0}{2}) & \beta(e^{-\frac{\nu_c r_0}{2}}+e^{\frac{\nu_c r_0}{2}}) & -\cos(\frac{\mu_r r_0}{2}) & -\frac{\gamma_r^2}{\mu_r^2 - \nu_r^2}\cos(\frac{\nu_r r_0}{2}) \\
        s_1\alpha\cos(\frac{\mu_c r_0}{2}) & s_2\beta(e^{-\frac{\nu_c r_0}{2}}+e^{\frac{\nu_c r_0}{2}}) & 0 & -\cos(\frac{\nu_r r_0}{2}) \\
        -D_{1c}\mu_c\alpha\cos(\frac{\mu_c r_0}{2}) & -D_{1c}\beta(-\nu_c e^{-\frac{\nu_c r_0}{2}}+\nu_c e^{\frac{\nu_c r_0}{2}}) & D_{1r}\mu_r\cos(\frac{\mu_r r_0}{2}) & D_{1r}\nu_r\frac{\gamma_r^2}{\mu_r^2 - \nu_r^2}\cos(\frac{\nu_r r_0}{2}) \\
        -D_{2c}\mu_c s_1\alpha\cos(\frac{\mu_c r_0}{2}) & -D_{2c}s_2\beta(-\nu_c e^{-\frac{\nu_c r_0}{2}}+\nu_c e^{\frac{\nu_c r_0}{2}}) & 0 & D_{2r}\nu_r \cos(\frac{\nu_r r_0}{2})
    \end{bmatrix}
\end{align*}
\normalsize
\begin{align*}
    \begin{split}
        \phi_{1r}^*(r) &= A_r e^{-\mu_r r} + \frac{\gamma_r^2}{\mu_r^2 - \nu_r^2}A_r e^{-\nu_r r} \\
        \phi_{2r}^*(r) &= B_r e^{-\nu_r r}
    \end{split}
\end{align*}
\end{frame}

\end{document}
