\section{讨论与总结}
\label{sec:discussion}

实际上$\det \pmb{M}(r_0)$的解有多个,如图\ref{fig:det_M_r_far}。

\begin{figure}[H]
    \centering
    \includesvg[width=0.6\linewidth]{figures/det_M_r_far.svg}
    \caption{$\det \pmb{M}(r)$}
    \label{fig:det_M_r_far}
\end{figure}

但第\ref{sec:calculation}节的计算中只取了$r_0$的正数解的最小值。
如此选择从物理上是合理的,对于无限大平板反应堆,反射层的大小与堆芯无关,在已知厚度为$r_0=38.715\si{cm}$的芯部可以达到临界时,仍增大堆芯厚度,会使得反应堆超临界。

若强行使用其他$r_0$的正数解,如$r_0=154.684\si{cm}$,则得到如图\ref{fig:flux_r_far}所示的中子通量密度,显然这个结果是不能成立的,因为通量密度需要恒正。

\begin{figure}[H]
    \centering
    \includesvg[width=0.6\linewidth]{figures/flux_r_far.svg}
    \caption{$\phi_1,\phi_2$计算结果,使用$r_0=154.684\si{cm}$,红色虚线为芯部与反射层的边界}
    \label{fig:flux_r_far}
\end{figure}

由图\ref{fig:flux_r}可以观察到,中子通量密度曲线在芯部与反射层交界面附近有一个凸起,
主要原因是反射层内的热中子吸收比芯部小,而其慢化能力比芯部强,由堆芯泄漏出来的快中子,由于反射层的慢化作用在反射层内大部分都变成了热中子。

在芯部快中子通量密度$\phi_{1c}$要比热中子通量密度$\phi_{2c}$大很多,且通量密度的形状有较大差异。
这主要是因为裂变产生的主要是快中子,而其中一部分在慢化为热中子的过程中已经泄漏出芯部或被吸收了。

图\ref{fig:flux_r_adjoint}中快中子伴随通量密度$\phi^*_{1c}$要比热中子伴随通量密度$\phi^*_{2c}$小很多,
说明对芯部引入慢中子吸收剂将引起更大的反应性变化,其中子价值更高。

本文从双群扩散方程出发,以一个典型压水堆的群常数为例,计算了双群中子通量密度。计算结果符合物理预期,加深了作者对反应堆物理的认知。
