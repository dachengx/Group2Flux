\section{计算结果}
\label{sec:results}

\subsection{临界堆芯宽度}

$\det \pmb{M}(r_0)=0$显然是一个没有求根公式的超越方程,我们通过数值方法\footnote{\url{https://docs.scipy.org/doc/scipy/reference/generated/scipy.optimize.minimize\_scalar.html}.},
首先对$\det \pmb{M}(r)$进行参数扫描,如图\ref{fig:det_M},

\begin{figure}[H]
    \centering
    \includesvg[width=0.6\linewidth]{figures/det_M_r.svg}
    \caption{$\det \pmb{M}(r)$的参数扫描,红色圆点处为$\left(r_0,\det \pmb{M}(r_0)\right)$}
    \label{fig:det_M}
\end{figure}

我们取$\det \pmb{M}(r_0)=0$所有正的解中的最小值,得到$r_0=38.715\si{cm}$。

此时$\pmb{M}$中的参数全部确定下来了,将$r_0$代回$\pmb{M}$。由于$\det \pmb{M}(r_0)=0$,方程\ref{eq:M_AB}有无穷多个非0解。
令$A_c=1$,将$\pmb{M}$表示为分块矩阵:

\begin{align}
    \pmb{M} &= \begin{bmatrix}
        M_{00} & M_{0,1\rightarrow 3} \\
        M_{1\rightarrow 3,0} & M_{1\rightarrow 3,1\rightarrow 3} \\
    \end{bmatrix}
\end{align}

则有

\begin{align}
    \begin{bmatrix}
        B_c \\
        A_r \\
        B_r
    \end{bmatrix} &= 
    -M_{1\rightarrow 3,1\rightarrow 3}^{-1}M_{1\rightarrow 3,0}
\end{align}

可以快速得到其余三个参数$\left[B_c,A_r,B_r\right]^{\mathrm{T}}$。

\subsection{双群通量}

以求得的$\left[A_c,B_c,A_r,B_r\right]^{\mathrm{T}}$计算$\phi_1,\phi_2$,得到结果如图\ref{fig:flux_r}。

\begin{figure}[H]
    \centering
    \includesvg[width=0.6\linewidth]{figures/flux_r.svg}
    \caption{$\phi_1,\phi_2$计算结果,红色虚线为芯部与反射层的边界}
    \label{fig:flux_r}
\end{figure}

红色虚线为芯部与反射层的边界,其左侧($r$较小一侧)为芯部,右侧为无限大反射层,可见中子通量密度在$r$非常大的时候趋近于$0$。

\subsection{双群伴随通量}
