\section{计算结果}
\label{sec:results}

\subsection{临界堆芯宽度}

$\det \pmb{M}(r_0)=0$显然是一个没有求根公式的超越方程,我们通过数值方法\footnote{\url{https://docs.scipy.org/doc/scipy/reference/generated/scipy.optimize.minimize\_scalar.html}.},
首先对$\det \pmb{M}(r)$进行参数扫描,如图\ref{fig:det_M},

\begin{figure}[H]
    \centering
    \includesvg[width=0.6\linewidth]{figures/det_M_r.svg}
    \caption{$\det \pmb{M}(r)$的参数扫描,红色圆点处为$\left(r_0,\det \pmb{M}(r_0)\right)$}
    \label{fig:det_M}
\end{figure}

我们取$\det \pmb{M}(r_0)=0$所有正的解中的最小值,得到$r_0=38.715\si{cm}$。

此时$\pmb{M}$中的参数全部确定下来了,将$r_0$代回$\pmb{M}$。由于$\det \pmb{M}(r_0)=0$,方程\ref{eq:M_AB}有无穷多个非0解。
令$A_c=1$,将$\pmb{M}$表示为分块矩阵:

\begin{align}
    \pmb{M} &= \begin{bmatrix}
        M_{00} & M_{0,1\rightarrow 3} \\
        M_{1\rightarrow 3,0} & M_{1\rightarrow 3,1\rightarrow 3} \\
    \end{bmatrix}
\end{align}

则有

\begin{align}
    \begin{bmatrix}
        B_c \\
        A_r \\
        B_r
    \end{bmatrix} &= 
    -M_{1\rightarrow 3,1\rightarrow 3}^{-1}M_{1\rightarrow 3,0}
\end{align}

可以快速得到其余三个参数$\left[B_c,A_r,B_r\right]^{\mathrm{T}}$。

\subsection{双群通量}

以求得的$\left[A_c,B_c,A_r,B_r\right]^{\mathrm{T}}$计算$\phi_1,\phi_2$,得到结果如图\ref{fig:flux_r}。

\begin{figure}[H]
    \centering
    \includesvg[width=0.6\linewidth]{figures/flux_r.svg}
    \caption{$\phi_1,\phi_2$计算结果,红色虚线为芯部与反射层的边界}
    \label{fig:flux_r}
\end{figure}

红色虚线为芯部与反射层的边界,其左侧($r$较小一侧)为芯部,右侧为无限大反射层,可见中子通量密度在$r$非常大的时候趋近于$0$。

\subsection{双群伴随通量}

对双群伴随通量的计算,等效于将群常数中的扩散系数$D_g$全部取相反数。
取反之后对\ref{eq:L_k_delta}中的$k',\delta$等参数没有影响,对\ref{eq:u12}也没有影响。所以芯部的中子通量密度函数参数化不变。

然而对于反射层的中子通量密度参数化,尤其是式\ref{eq:mu_nu_gamma},不再成立。为伴随通量更新反射层的中子通量密度:

\begin{align}
    \label{eq:mu_nu_gamma_s}
    \begin{split}
        \mu_r &= \sqrt{-\frac{\Sigma_{rr}}{D_{1r}}} \\
        \nu_r &= \sqrt{-\frac{\Sigma_{a2r}}{D_{2r}}} \\
        \gamma_r &= \sqrt{-\frac{\Sigma_{1\rightarrow 2,r}}{D_{2r}}} \\
    \end{split}
\end{align}
\begin{align}
    \label{eq:para_phi_r_s}
    \begin{split}
        \phi_{1r}^*(r) &= A_r \cos(\mu_r r) \\
        \phi_{2r}^*(r) &= \frac{\gamma_r^2}{\nu_r^2 - \mu_r^2}A_r \cos(\mu_r r) + B_r \cos(\nu_r r)
    \end{split}
\end{align}

其中式\ref{eq:para_phi_r_s}中为保持$\phi_{1r}^*,\phi_{2r}^*$的偶函数性质,均只取余弦。
$C_r$的表达式不变:

\begin{align}
    C_r &= \frac{\gamma_r^2}{\nu_r^2 - \mu_r^2}A_r
\end{align}

同时更新对$\pmb{M}^*(r_0)$的计算:

\begin{align}
    \pmb{M}^*(r_0) &= 
    \begin{bmatrix}
        \alpha\cos(\frac{\mu_c r_0}{2}) & \beta(e^{-\frac{\nu_c r_0}{2}}+e^{\frac{\nu_c r_0}{2}}) & -\cos(\frac{\mu_r r_0}{2}) & 0 \\
        s_1\alpha\cos(\frac{\mu_c r_0}{2}) & s_2\beta(e^{-\frac{\nu_c r_0}{2}}+e^{\frac{\nu_c r_0}{2}}) & -\frac{\gamma_r^2}{\nu_r^2 - \mu_r^2}\cos(\frac{\mu_r r_0}{2}) & -\cos(\frac{\nu_r r_0}{2}) \\
        -D_{1r}\mu_c\alpha\cos(\frac{\mu_c r_0}{2}) & -D_{1r}\beta(-\nu_c e^{-\frac{\nu_c r_0}{2}}+\nu_c e^{\frac{\nu_c r_0}{2}}) & D_{2r}\mu_r \cos(\frac{\mu_r r_0}{2}) & 0 \\
        -D_{1r}\mu_c s_1\alpha\cos(\frac{\mu_c r_0}{2}) & -D_{1r}s_2\beta(-\nu_c e^{-\frac{\nu_c r_0}{2}}+\nu_c e^{\frac{\nu_c r_0}{2}}) & D_{2r}\mu_r\frac{\gamma_r^2}{\nu_r^2 - \mu_r^2}\cos(\frac{\mu_r r_0}{2}) & D_{2r}\nu_r \cos(\frac{\nu_r r_0}{2})
    \end{bmatrix}
\end{align}

类似地,对$\det \pmb{M}(r)$进行参数扫描,如图\ref{fig:det_M_adjoint},

\begin{figure}[H]
    \centering
    \includesvg[width=0.6\linewidth]{figures/det_M_r_adjoint.svg}
    \caption{$\det \pmb{M}^*(r)$的参数扫描,红色圆点处为$\left(r_0,\det \pmb{M}^*(r_0)\right)$}
    \label{fig:det_M_adjoint}
\end{figure}

以求得的$\left[A_c,B_c,A_r,B_r\right]^{\mathrm{T}}$计算$\phi_1^*,\phi_2^*$,得到结果如图\ref{fig:flux_r_adjoint}。

\begin{figure}[H]
    \centering
    \includesvg[width=0.6\linewidth]{figures/flux_r_adjoint.svg}
    \caption{$\phi_1^*,\phi_2^*$计算结果,红色虚线为芯部与反射层的边界}
    \label{fig:flux_r_adjoint}
\end{figure}
