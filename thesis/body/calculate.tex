\section{计算过程}
\label{sec:calculation}

共有两组函数矢量待解,其中$\phi_{1c}(r)$

\subsection{芯部通量参数化}

为了求解方程\ref{eq:A_vector_phi},将矩阵$\pmb{A}$进行特征值分解,首先求出特征值$\mu^2,-\nu_2$,如式\ref{eq:mu_nu_c}。

\begin{align}
    \label{eq:mu_nu_c}
    \begin{split}
        \mu^2 &= \frac{1}{2}\left[-\left(\frac{1}{L^2}+\frac{1}{\tau}\right)+\sqrt{\left(\frac{1}{L^2}+\frac{1}{\tau}\right)^2+\frac{4(\delta-1)}{\tau L^2}}\right] \\
        \nu^2 &= \frac{1}{2}\left[\left(\frac{1}{L^2}+\frac{1}{\tau}\right)+\sqrt{\left(\frac{1}{L^2}+\frac{1}{\tau}\right)^2+\frac{4(\delta-1)}{\tau L^2}}\right]
    \end{split}
\end{align}

其中

\begin{align}
    \begin{split}
        L^2 = \frac{D_2}{\Sigma_{a2}},\ \tau &= \frac{D_1}{\Sigma_r - \frac{\nu\Sigma_{f1}}{k_\mathrm{eff}}},\ \delta = \frac{k'-\frac{\nu\Sigma_{f1}}{\Sigma_r}}{k_\mathrm{eff}-\frac{\nu\Sigma_{f1}}{\Sigma_r}} \\
        k' &= \frac{\nu\Sigma_{f1}+\nu\Sigma_{f2}\frac{\Sigma_{1\rightarrow 2}}{\Sigma_{a2}}}{\Sigma_r}
    \end{split}
\end{align}

当反应堆临界时有效增值系数$k_\mathrm{eff}=1$,此时$\pmb{A}$和$\mu^2,\nu^2$均为常量。由群常数可计算得到,$\mu^2\neq-\nu^2$。
同时也可以求出$\mu^2,-\nu^2$对应的特征单位向量$\pmb{u}_1,\pmb{u}_2$,如式\ref{eq:u12}。

\begin{align}
    \label{eq:u12}
    \begin{split}
        \pmb{u}_1 = \begin{bmatrix}
            \alpha \\
            s_1\alpha
        \end{bmatrix},\ 
        \pmb{u}_2 = \begin{bmatrix}
            \beta \\
            s_2\beta
        \end{bmatrix}
    \end{split}
\end{align}

其中$s_{1,2}$为群常数和$k_\mathrm{eff}$的函数,$\alpha,\beta$为归一化常数,

\begin{align}
    s_1 &= \frac{\frac{\Sigma_{1\rightarrow 2}}{D_2}}{\mu^2 + \frac{1}{L^2}},\ s_2 = \frac{\frac{\Sigma_{1\rightarrow 2}}{D_2}}{\frac{1}{L^2}-\nu^2} \\
    \alpha &= \left(1+s_1^2\right)^{-1/2},\ \beta = \left(1+s_2^2\right)^{-1/2}
\end{align}

此时由单位特征向量$\pmb{u}_1,\pmb{u}_2$构成的转移矩阵$\pmb{U}$如式\ref{eq:U}。

\begin{align}
    \label{eq:U}
    \pmb{U} &= \begin{bmatrix}
        \pmb{u}_1, \pmb{u}_2
    \end{bmatrix}
\end{align}

有如下特征值分解:

\begin{align}
    \pmb{A}\pmb{U} &= \pmb{U}\pmb{\Lambda}
\end{align}

其中$\pmb{\Lambda}$为对角阵。

\begin{align}
    \pmb{\Lambda} &= \begin{bmatrix}
        \mu^2 & \\
         & -\nu^2
    \end{bmatrix}.
\end{align}

引入函数向量变换

\begin{align}
    \label{eq:phi_psi}
    \pmb{\Phi}(r) &= \pmb{U}\pmb{\Psi}(r)
\end{align}

式中$\pmb{\Psi}(r)$为新的函数矢量,设其两个函数分别为$\psi_{1,2}(r)$,则有方程如式\ref{eq:psi_12}。

\begin{align}
    \pmb{\Psi} = \begin{bmatrix}
        \psi_1(r) \\
        \psi_2(r)
    \end{bmatrix}
    \label{eq:psi_12}
    \pmb{U}\left\{\pmb{\nabla}^2\pmb{\Psi}(r)+\pmb{\Lambda}\pmb{\Psi}(r)\right\} &= \pmb{0}
\end{align}

因$\pmb{U}$中的元素不全为$0$,则必然要求如下方程恒成立:

\begin{align}
    \nabla^2\psi_1(r) + \mu^2\psi_1(r) &= 0 \\
    \nabla^2\psi_2(r) - \nu^2\psi_2(r) &= 0
\end{align}

这两个常系数常微分方程的通解为:

\begin{align}
    \psi_{1c}(r) &= A_c\cos{\mu_c r} + A'_c\sin{\mu_c r} \\
    \psi_{2c}(r) &= B_c e^{-\nu_c r} + B'_c e^{\nu_c r}
\end{align}

其中$A_c,B_c,A'_c,B'_c$均为待求常数,为与反射层中物理量做区别,以下对芯部物理量使用下标$c$,对反射层物理量使用下标$r$。

由于反应堆几何关于$r=0$对称,则$A'_c=0$,亦即奇函数贡献为$0$。
同时$\psi_{2c}(r)$为偶函数,在$r=0$时应连续且可导,则有$B_c=B'_c$,最终芯部的参数化为式\ref{eq:para_psi}。

\begin{align}
    \label{eq:para_psi}
    \begin{split}
        \psi_{1c}(r) &= A_c\cos{\mu_c r} \\
        \psi_{2c}(r) &= B_c\left(e^{-\nu_c r} + e^{\nu_c r}\right)
    \end{split}
\end{align}

代回函数向量变换\ref{eq:phi_psi},得到$r$空间上的中子通量密度参数化函数,如式\ref{eq:para_phi}。

\begin{align}
    \label{eq:para_phi}
    \begin{split}
        \phi_{1c}(r) &= \alpha\psi_{1c}(r) + \beta\psi_{2c}(r) = \alpha A_c\cos{\mu_c r} + \beta B_c\left(e^{-\nu_c r} + e^{\nu_c r}\right) \\
        \phi_{2c}(r) &= s_1\alpha\psi_{1c}(r) + s_2\beta\psi_{2c}(r) = s_1\alpha A_c\cos{\mu_c r} + s_2\beta B_c\left(e^{-\nu_c r} + e^{\nu_c r}\right)
    \end{split}
\end{align}

待求参数为$A_c,B_c$,公式适用于$r \leq r_0/2$的区间。

\subsection{反射层通量参数化}

对于反射层中的中子通量密度,由于$\nu\Sigma_f=0$,没有中子裂变谱的贡献,处理方式较芯部简单。首先列出无中子源的反射层双群扩散方程,如式\ref{eq:two_reflect}。

\begin{align}
    \label{eq:two_reflect}
    \begin{split}
        -D_{1r}\nabla^2\phi_{1r}(r) + \Sigma_{rr}\phi_{1r}(r) &= 0 \\
        -D_{2r}\nabla^2\phi_{2r}(r) + \Sigma_{a2r}\phi_{2r}(r) &= \Sigma_{1\rightarrow 2,r}\phi_{1r}(r)
    \end{split}
\end{align}

其中下标$r$表示反射层中的物理量。可见反射层中的快群通量$\phi_{1r}(r)$不与热群通量$\phi_{2r}(r)$直接耦合,可以先求出$\phi_{1r}(r)$再推导出$\phi_{2r}(r)$。

定义常数$\mu_r,\nu_r,\gamma_r$,他们都是群常数的函数。

\begin{align}
    \mu_r &= \sqrt{\frac{\Sigma_{rr}}{D_{1r}}} \\
    \nu_r &= \sqrt{\frac{\Sigma_{a2r}}{D_{2r}}} \\
    \gamma_r &= \sqrt{\frac{\Sigma_{1\rightarrow 2,r}}{D_{2r}}}
\end{align}

则反射层中的中子通量密度$\phi_{1r},\phi_{2r}$参数化为式\ref{eq:para_phi_rC}。
其中考虑到反射层无限大,且中子通量密度在无限远处为$0$,$\phi_{1r}(r)$直接舍弃了随着$r$的增大趋于无穷的$e^{\mu_r r}$项;$A_r,B_r,C_r$均为待求参数。

\begin{align}
    \label{eq:para_phi_rC}
    \begin{split}
        \phi_{1r}(r) &= A_r e^{-\mu_r r} \\
        \phi_{2r}(r) &= B_r e^{-\nu_r r} + C_r e^{-\mu_r r}
    \end{split}
\end{align}

将$\phi_{1,2r}(r)$带入\ref{eq:two_reflect},

\begin{align}
    \mu_r^2 C_r e^{-\mu_r r} - \nu_r^2 C_r e^{-\mu_r r} + \gamma_r A_r e^{-\mu_r r} &= 0
\end{align}

得到$C_r$与$A_r,B_r$的关系,

\begin{align}
    C_r &= \frac{\gamma_r^2}{\nu_r^2 - \mu_r^2}A_r
\end{align}

至此对反射层中的中子通量密度的参数化完成,参数为$A_r,B_r$,公式适用于$r \geq r_0/2$的区间。

\begin{align}
    \label{eq:para_phi_r}
    \begin{split}
        \phi_{1r}(r) &= A_r e^{-\mu_r r} \\
        \phi_{2r}(r) &= \frac{\gamma_r^2}{\nu_r^2 - \mu_r^2}A_r e^{-\mu_r r} + B_r e^{-\nu_r r}
    \end{split}
\end{align}

\subsection{边界条件}

为了保证中子通量密度和其导数的连续性(一阶可导),我们要求连续性边界条件(式\ref{eq:continue})成立。

\begin{align}
    \label{eq:continue}
    \begin{split}
        \phi_{jc}(r_0 / 2) &= \phi_{jr}(r_0 / 2) \\
        D_{jc}\phi_{jc}(r_0 / 2) &= D_{jr}\phi_{jr}(r_0 / 2) \\
        j = 1,2
    \end{split}
\end{align}

将\ref{eq:para_phi}和\ref{eq:para_phi_r}带入连续性边界条件,经过整理,最终得到$A_c,B_c,A_r,B_r$的线性方程为\ref{eq:M_AB},

\begin{align}
    \label{eq:M_AB}
    \pmb{M}(r_0) \cdot \begin{bmatrix}
        A_c \\
        B_c \\
        A_r \\
        B_r
    \end{bmatrix} &= \pmb{0}
\end{align}
\begin{align}
    \pmb{M}(r_0) &= 
    \begin{bmatrix}
        \alpha\cos(\frac{\mu_c r_0}{2}) & \beta(e^{-\frac{\nu_c r_0}{2}}+e^{\frac{\nu_c r_0}{2}}) & -e^{-\frac{\mu_r r_0}{2}} & 0 \\
        s_1\alpha\cos(\frac{\mu_c r_0}{2}) & s_2\beta(e^{-\frac{\nu_c r_0}{2}}+e^{\frac{\nu_c r_0}{2}}) & -\frac{\gamma_r^2}{\nu_r^2 - \mu_r^2}e^{-\frac{\mu_r r_0}{2}} & -e^{-\frac{\nu_r r_0}{2}} \\
        -D_{1r}\mu_c\alpha\cos(\frac{\mu_c r_0}{2}) & -D_{1r}\beta(-\nu_c e^{-\frac{\nu_c r_0}{2}}+\nu_c e^{\frac{\nu_c r_0}{2}}) & D_{2r}\mu_r e^{-\frac{\mu_r r_0}{2}} & 0 \\
        -D_{1r}\mu_c s_1\alpha\cos(\frac{\mu_c r_0}{2}) & -D_{1r}s_2\beta(-\nu_c e^{-\frac{\nu_c r_0}{2}}+\nu_c e^{\frac{\nu_c r_0}{2}}) & D_{2r}\mu_r\frac{\gamma_r^2}{\nu_r^2 - \mu_r^2}e^{-\frac{\mu_r r_0}{2}} & D_{2r}\nu_r e^{-\frac{\nu_r r_0}{2}}
    \end{bmatrix}
\end{align}

其中较复杂的$4\times4$矩阵$\pmb{M}$为$r_0$的函数,$\left[A_c,B_c,A_r,B_r\right]^{\mathrm{T}}$有非$0$非平凡解的条件为:

\begin{equation}
    \det \pmb{M}(r_0) = 0
\end{equation}

在求出临界宽度$r_0$后我们可以通过求解方程\ref{eq:M_AB}得到$\left[A_c,B_c,A_r,B_r\right]^{\mathrm{T}}$的通解。
相关计算结果将在第\ref{sec:results}节中展示。
